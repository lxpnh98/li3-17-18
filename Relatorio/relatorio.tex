\documentclass[10pt]{report}

\usepackage[a4paper]{geometry}
\usepackage[utf8]{inputenc}
\usepackage[portuguese]{babel}

\newcommand\tab[1][0.5cm]{\hspace*{#1}}


\title{Projeto de Laboratórios de Informática III\\Grupo 25	}
\author{Alexandre Mendonça Pinho (a82441) \and Tiago Martins Pinheiro (a82491) \and Joel Filipe Esteves Gama (a82202)}
\date{\today}

\begin{document}
\maketitle

\tableofcontents
\chapter{Introdução}
\label{sec:intro}

\tab Este relatório apresenta uma explicação da primeira parte do projeto da disciplina de Laboratórios de Informática III, do 2º ano do Mestrado Integrado em Engenharia Informática da Universidade do Minho, que toma a forma de um projeto a ser desenvolvido na linguagem de programação imperativa \textit{C}.

Como projeto de avaliação é nos proposto responder a onze interrogações sobre processamento de dados do Stack Overflow. Como forma de resposta desenvolvemos as estruturas de dados, tivemos em consideração a abstração e modulação dos dados e também usamos os algoritmos que nos pareceram mais adequados para conseguir um bom resultado nos tempos de resposta de cada interrogação.

Neste relatório apresentamos a descrição do problema a resolver, assim como explicamos os tipos de dados, as estruturas usadas e as estratégias de resposta e melhoria dos algoritmos de resposta às interrogações.
\end{document}