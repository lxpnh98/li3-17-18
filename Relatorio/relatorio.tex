\documentclass[10pt]{report}

\usepackage[a4paper]{geometry}
\usepackage[utf8]{inputenc}
\usepackage[portuguese]{babel}

\newcommand\tab[1][0.5cm]{\hspace*{#1}}


\title{Projeto de Laboratórios de Informática III\\Grupo 25	}
\author{Alexandre Mendonça Pinho (a82441) \and Joel Filipe Esteves Gama (a82202) \and Tiago Martins Pinheiro (a82491)}
\date{\today}

\begin{document}
\maketitle

\tableofcontents
\chapter{Introdução}
\label{sec:intro}

\tab Este relatório apresenta uma explicação da primeira fase do projeto da disciplina de Laboratórios de Informática III, do 2º ano do Mestrado Integrado em Engenharia Informática da Universidade do Minho, que toma a forma de um projeto a ser desenvolvido na linguagem de programação imperativa \textit{C}.

Como projeto de avaliação é nos proposto responder a onze interrogações sobre processamento de dados do Stack Overflow. Como forma de resposta desenvolvemos as estruturas de dados, tivemos em consideração a abstração e modulação dos dados e também usamos os algoritmos que nos pareceram mais adequados para conseguir um bom resultado nos tempos de resposta de cada interrogação.

Neste relatório apresentamos a descrição do problema a resolver, assim como explicamos os tipos de dados, as estruturas usadas e as estratégias de resposta e melhoria dos algoritmos de resposta às interrogações.
\chapter{Descrição do Problema}
\label{sec:problema}

\tab O projeto está dividido em duas fases, sendo o objetivo desta primeira fase responsde às onze interrogações utilizando a linguagem de progamação \textit{C}. Para responder às onze interrogações foi necessário a escrita de várias funções, fazendo com que o projeto seja maioritariamente escrita de código, sendo também necessário ter conhecimentos de debugging, algoritmos de procura e biblioteca XML.
Em cada interrogação são pretendidos resultados distintos. Utilizando os tipos já disponibilizados pelos professores tivemos de implementar as funções de resposta, assim como as funções auxiliares para cada interrogação.

\iffalse
\chapter{Conceção da Solução}
\section{Tipos concretos de dados}

\section{Estruturas de dados usadas}

\section{Modulação funcional}

\section{Abstração de dados}

\section{Estratégias seguidas em cada uma das interrogações}

\section{Estratégias para melhoramento de desempenho}
\fi

\chapter{Conclusão}
\label{sec:conclusao}

\end{document}